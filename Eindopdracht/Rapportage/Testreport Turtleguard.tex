%!TEX TS-program = xelatex
%!TEX encoding = UTF-8 Unicode
\documentclass[a4paper]{report}
%\usepackage[date=short,backend=biber]{apa}
\usepackage[hidelinks]{hyperref}
\usepackage{apacite}
\usepackage[dutch]{babel}
\usepackage[a4paper, left=1in, right=1in, top=1in, bottom=.8in]{geometry}
\usepackage[utf8]{inputenc}
\usepackage{fancyhdr}
\usepackage{nameref}
\usepackage{helvet}
\usepackage{titlesec}
\usepackage{geometry}
\usepackage{ragged2e}
\usepackage{graphicx}
\usepackage{etoolbox}
\usepackage{listings}
\usepackage{xspace}
\usepackage[table]{xcolor}
\usepackage{nameref}
\usepackage{tcolorbox}
\usepackage{textcomp}
\usepackage{colortbl}
\usepackage{glossaries}
\usepackage{tabularx}
\usepackage{float}
\usepackage{pgffor}
\usepackage{listings}

\definecolor{bg}{rgb}{0.1, 0.1, 0.1}

% Styling
\renewcommand{\rmdefault}{\sfdefault}
\pagestyle{fancy}
\patchcmd{\chapter}{\thispagestyle{plain}}{\thispagestyle{fancy}}{}{}

\fancyhf{}
\fancyhead[L]{ \turtleguard }
\fancyhead[R]{ Opdracht 4 (ATP) }
\fancyfoot[R]{\thepage}

\titleformat{\chapter}[hang]
{\normalfont\huge\bfseries}{\thechapter.}{10pt}{\huge}
\titlespacing{\chapter}{0pt}{-30pt}{20pt}

\setlength{\parindent}{0.2em}

\textwidth=400pt
\geometry{
    left=25mm
}

\renewcommand{\contentsname}{Inhoudsopgave}
%\RaggedRight % Don't 'block-justify' text

\definecolor{codegreen}{rgb}{0,0.6,0}
\definecolor{codegray}{rgb}{0.5,0.5,0.5}
\definecolor{codepurple}{rgb}{0.58,0,0.82}
\definecolor{backcolour}{rgb}{0.95,0.95,0.92}

\lstdefinestyle{mystyle}{
    backgroundcolor=\color{backcolour},   
    commentstyle=\color{codegreen},
    keywordstyle=\color{magenta},
    numberstyle=\tiny\color{codegray},
    stringstyle=\color{codepurple},
    basicstyle=\ttfamily\footnotesize,
    breakatwhitespace=false,         
    breaklines=true,                 
    captionpos=b,                    
    keepspaces=true,                 
    numbers=left,                    
    numbersep=5pt,                  
    showspaces=false,                
    showstringspaces=false,
    showtabs=false,                  
    tabsize=2
}

\lstset{style=mystyle}



% Commands
\newcommand{\teambox}{
  \begin{tcolorbox}[hbox, colback=blue!5!white,colframe=blue!75!black,
    left=.1mm, right=.1mm, top=.1mm, bottom=.1mm, fontupper=\scriptsize\sffamily]
    Team Keuze
  \end{tcolorbox}
}

\newcommand{\personalbox}{
  \begin{tcolorbox}[hbox, colback=green!5!white,colframe=green!75!black,
    left=.1mm, right=.1mm, top=.1mm, bottom=.1mm, fontupper=\scriptsize\sffamily]
    Persoonlijke Keuze
  \end{tcolorbox}
}
\newcommand{\teamchoice}[1]{
  \section[ #1 ]{#1~\mbox{\teambox}}
}

\newcommand{\personalchoice}[1]{
  \section[ #1 ]{#1~\mbox{\personalbox}}
}
\newcommand{\turtleguard}{\mbox{TurtleGuard\texttrademark}\xspace}


% Document
\begin{document}


% Title Page
\begin{titlepage}
    \begin{center}
        \vspace*{.9cm}
        \Huge
        \textbf{ Test Rapportage voor \turtleguard }\\
        \vspace{0.2cm}
        \small\textit{"Hét ultieme schild-bad voor jouw schildpad"}

        \normalsize


        
        \includegraphics[width=0.7\textwidth]{Images/turtleguard.png}
        \vspace{1cm}
        \Large\\
        \textbf{Mede mogelijk gemaakt door} \\
        \includegraphics[width=0.2\textwidth]{Images/logouni.png}


        \vfill
      \end{center}
        \textbf{Student:} Vincent van Setten - 1734729 \\
        \textbf{Opdrachtgever:} HU University of Applied Sciences\\
        \textbf{Datum:} \today \\
        \vspace{2cm}
\end{titlepage}


% ToC
\tableofcontents


\clearpage  % End of the page

\chapter{Inleiding}
Binnen het vak ATP op de Hogeschool Utrecht, onderdeel van het bredere Innovation project, zal ik uiteindelijk een regelsysteem moeten simuleren.
Het regelsysteem zal geprogrammeerd worden in Python, volgens het Functionele paradigma, en zal in dit document verder aangeduid worden als \turtleguard.
Dit document dient korte toelichting van de uitgevoerde tests, de resultaten hiervan en het nut hiervan.

\chapter{Unit Tests}
\section{Uitvoering}
Voor het uitvoeren van deze(en alle andere tests) kun je simpelweg 'make test' uitvoeren. Dit zorgt dat de tests worden uitgevoerd en gecontroleerd, door middel van de python library 'unittest'.
\section{Uitgevoerde Tests}
Er worden verschillende tests uitgevoerd welke controleerde of de controlWaterTemperature functie goed werkte.
Dit werd gedaan door vooraf bepaalde temperatuur waarden mee te geven en te controleren of de teruggekregen actuator acties klopte.
Als het water bijvoorbeeld te koud is, moet de verwarming aan en de koeler uit. Kijk in het test.py document voor de code.

\section{Resultaten}
Alle tests waren succesvol.

\section{Conclusie}
Volgens de test klopt de werking van de controlWaterTemperature functie. Dit betekent dat de keuze om de actuators van de verwarming en koeler klopt.
Hiermee kunnen we zeker zijn dat als de waardes goed worden doorgegeven, de temperatuur juist wordt geregeld.
Op deze manier zijn we dus zeker dat de schildpad in veilige temperaturen gehouden wordt.

\chapter{Integratie Tests}
\section{Uitvoering}
Voor het uitvoeren van deze(en alle andere tests) kun je simpelweg 'make test' uitvoeren. Dit zorgt dat de tests worden uitgevoerd en gecontroleerd, door middel van de python library 'unittest'.
\section{Uitgevoerde Tests}
Er is een enkele integratie test uitgevoerd dat controleert of, gegeven vooraf bepaalde sensorwaardes, de juiste actuator functies worden aangeroepen(zoals verwarming aan).

\section{Resultaten}
Deze test was succesvol. 

\section{Conclusie}
Doordat dit goed gaat, kunnen we verifiëren dat als de sensor waardes kloppen, de hele main\_loop in een enkele iteratie goed gaat.
Dit zorgt er dus voor dat alle keuzes kloppen en de juiste actuator functies aan de juiste keuzes zijn gekoppeld.
Kortweg betekent dit dat de hele regellogica klopt, gegeven de juiste waarden. Hiermee kunnen we dus zeker zijn dat het systeem voorspelbaar reageert naar verwachtingen.

\chapter{Systeemtest}
\section{Uitvoering}
Voor het uitvoeren van deze(en alle andere tests) kun je simpelweg 'make test' uitvoeren. Dit zorgt dat de tests worden uitgevoerd en gecontroleerd, door middel van de python library 'unittest'.
\section{Uitgevoerde Tests}
Er is een enkele systeem test uitgevoerd.
Deze systeemtest maakt gebruik van een simulator(zie plant.py) die een hele plant simuleert. 
Door gesimuleerde sensoren en actuatoren mee te geven, reageert de plant op de keuzes van het regelsysteem.
De plant verandert per iteratie. Deze iteratie staat gelijk aan ongeveer een minuut in de echte wereld.
De systeemtest voert 10.000 iteraties uit op deze simulatie(wat gelijk staat aan 10.000 minuten, oftewel bijna 7 dagen).
Aan het eind van de simulatie wordt gecontroleerd of de gesimuleerde schildpad nog leeft. 
\section{Resultaten}
Deze test was succesvol(de schildpad was nog in leven).

\section{Conclusie}
Gedurende een (gesimuleerde) week heeft het regelsysteem de juiste keuzes gemaakt, gebaseerd op een gesimuleerde plant.
Dit betekent dat met een grote reeks aan verschillende scenario's het regelsysteem de juiste acties heeft uitgevoerd en de schildpad gezond heeft gehouden.
Hiermee is het systeem voorspelbaar en veilig.

\end{document}