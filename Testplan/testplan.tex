%!TEX TS-program = xelatex
%!TEX encoding = UTF-8 Unicode
\documentclass[a4paper]{report}
%\usepackage[date=short,backend=biber]{apa}
\usepackage{hyperref}
\usepackage{apacite}
\usepackage[dutch]{babel}
\usepackage[a4paper, left=1in, right=1in, top=1in, bottom=.8in]{geometry}
\usepackage[utf8]{inputenc}
\usepackage{fancyhdr}
\usepackage{nameref}
\usepackage{helvet}
\usepackage{titlesec}
\usepackage{geometry}
\usepackage{ragged2e}
\usepackage{graphicx}
\usepackage{etoolbox}
\usepackage{listings}
\usepackage{xspace}
\usepackage{xcolor}
\usepackage{nameref}
\usepackage{tcolorbox}
\usepackage{textcomp}


% Styling
\renewcommand{\rmdefault}{\sfdefault}
\pagestyle{fancy}
\patchcmd{\chapter}{\thispagestyle{plain}}{\thispagestyle{fancy}}{}{}

\fancyhf{}
\fancyhead[L]{ \turtleguard }
\fancyhead[R]{ Projectplan }
\fancyfoot[R]{\thepage}

\titleformat{\chapter}[hang]
{\normalfont\huge\bfseries}{\thechapter.}{10pt}{\huge}
\titlespacing{\chapter}{0pt}{-30pt}{20pt}

\setlength{\parindent}{0.2em}

\textwidth=400pt
\geometry{
    left=25mm
}

\renewcommand{\contentsname}{Inhoudsopgave}
\RaggedRight % Don't 'block-justify' text

\definecolor{codegreen}{rgb}{0,0.6,0}
\definecolor{codegray}{rgb}{0.5,0.5,0.5}
\definecolor{codepurple}{rgb}{0.58,0,0.82}
\definecolor{backcolour}{rgb}{0.95,0.95,0.92}

\lstdefinestyle{mystyle}{
    backgroundcolor=\color{backcolour},   
    commentstyle=\color{codegreen},
    keywordstyle=\color{magenta},
    numberstyle=\tiny\color{codegray},
    stringstyle=\color{codepurple},
    basicstyle=\ttfamily\footnotesize,
    breakatwhitespace=false,         
    breaklines=true,                 
    captionpos=b,                    
    keepspaces=true,                 
    numbers=left,                    
    numbersep=5pt,                  
    showspaces=false,                
    showstringspaces=false,
    showtabs=false,                  
    tabsize=2
}

\lstset{style=mystyle}

% Commands
\newcommand{\teambox}{
  \begin{tcolorbox}[hbox, colback=blue!5!white,colframe=blue!75!black,
    left=.1mm, right=.1mm, top=.1mm, bottom=.1mm, fontupper=\scriptsize\sffamily]
    Team Keuze
  \end{tcolorbox}
}

\newcommand{\personalbox}{
  \begin{tcolorbox}[hbox, colback=green!5!white,colframe=green!75!black,
    left=.1mm, right=.1mm, top=.1mm, bottom=.1mm, fontupper=\scriptsize\sffamily]
    Persoonlijke Keuze
  \end{tcolorbox}
}
\newcommand{\teamchoice}[1]{
  \section[ #1 ]{#1~\mbox{\teambox}}
}

\newcommand{\personalchoice}[1]{
  \section[ #1 ]{#1~\mbox{\personalbox}}
}
\newcommand{\turtleguard}{\mbox{TurtleGuard\texttrademark}\xspace}

% Document
\begin{document}


% Title Page
\begin{titlepage}
    \begin{center}
        \vspace*{.9cm}
        \Huge
        \textbf{ Projectplan \turtleguard }\\
        \vspace{0.2cm}
        \small\textit{"Hét ultieme schild-bad voor jouw schildpad"}

        \normalsize


        
        \includegraphics[width=0.7\textwidth]{Images/turtleguard.png}
        \vspace{1cm}
        \Large\\
        \textbf{Mede mogelijk gemaakt door} \\
        \includegraphics[width=0.2\textwidth]{Images/logouni.png}


        \vfill
      \end{center}
        \textbf{Student:} Vincent van Setten - 1734729 \\
        \textbf{Opdrachtgever:} HU University of Applied Sciences\\
        \textbf{Datum:} \today \\
        \vspace{2cm}
\end{titlepage}


% ToC
\tableofcontents

\chapter*{Versiebeheer}
\thispagestyle{empty}  % Removes header and footer from this specific page
\begin{table}[h]
    \centering
    \begin{tabular}{|c|c|c|p{5cm}|}
        \hline
        Versie & Datum      & Changes Made  \\
        \hline
        1.0    & 2023-09-10 & Eerste Versie \\
        \hline
    \end{tabular}
    \caption{Versiebeheer}
\end{table}
\clearpage  % End of the page

\chapter{Algemene Informatie}
\section{Introductie}
Dit document dient als project- en testplan voor het \turtleguard systeem. Het \turtleguard systeem wordt verder in detail besproken in hoofdstuk \ref{section-product}.
Verdere hoofdstukken beschrijven de architectuur van het product, de eisen aan het product en de tests die uitgevoerd zullen worden.
Het uiteindelijke doel van dit document is het beschrijven van een regelsysteem, in dit geval \turtleguard, welke ten minste twee sensoren en actuatoren bevat. 

\section{Productomschrijving}
\label{section-product}
Huisdieren, we zijn er (bijna) allemaal gek op. Vanzelfsprekend is het dus dat we het onze vertrouwde gezelschap zo comfortabel mogelijk willen maken.
Bij de meeste huisdieren is het relatief simpel, \mbox{zoals} bij een kat of hond. Zorg voor eten, drinken en aandacht en dan zijn ze vaak wel tevreden.
\par \smallskip
Helaas is het niet bij alle huisdieren zo eenvoudig. 
Zo heeft bijvoorbeeld een waterschildpad best strenge eisen aan het water: naast het voeren, moeten bijvoorbeeld ook de temperatuur en de \mbox{pH-waarde} van het water precies goed zijn.  
Als mens kunnen wij dit niet altijd even eenvoudig controleren. \par \smallskip 

Daarom stellen wij voor: \turtleguard.
Dit systeem hanteert een specifieke temperatuur voor uw \mbox{waterschildpad} en laat u weten als de waterkwaliteit niet aan de eisen voldoet.
Met \turtleguard hoeft u zich nooit meer druk te maken of uw schildpad zich wel comfortabel voelt in zijn schild.

\section{Kwaliteitscriteria}

\section{Begrippen}

\chapter{Systeem Architectuur}
\section{Hardware}
\section{Software}
\chapter{Testplan}
\section{Unit Tests}
\section{Integratietests}
\section{Systeemtests}
\section{Risicoanalyse}


\chapter{Bibliografie}
\nocite{*} % This includes all entries from the .bib file, even if they're not cited in the document
\begingroup
\renewcommand{\chapter}[2]{} % Removes the 'Chapter' heading
\renewcommand{\addcontentsline}[3]{} % Prevents adding this specific entry to TOC
\bibliographystyle{apacite}
\bibliography{bronnen}
\endgroup

\chapter{Bijlagen}

\end{document}